\chapter{Stochastic Processes and Diffusion}\label{sec:sp}
Stochastic processes model the evolution of random variables over time. Diffusion-based generative models rely heavily on concepts from Brownian motion and stochastic differential equations.

\section{Overview of Stochastic Processes}
A stochastic process $\{X_t\}_{t\ge0}$ is a collection of random variables indexed by time. Processes may be discrete- or continuous-time and can have state spaces ranging from finite sets to $\mathbb{R}^d$. Key properties include stationarity and memory. Understanding these notions prepares us for Markov models used in diffusion and autoregressive techniques.

\section{Markov Chains}
Discrete-time Markov chains satisfy $\mathbb{P}(X_{t+1}=j\mid X_t=i,\ldots)=P_{ij}$ where $P$ is a transition matrix with rows summing to one. The chain is \emph{irreducible} if every state communicates and \emph{aperiodic} if returns occur at irregular times. The stationary distribution $\pi$ solves $\pi^T P=\pi^T$. Markov chains model sampling procedures in MCMC and are discrete analogues of diffusion processes.

\section{Brownian Motion}
\subsection{Definition and Properties}
\begin{definition}[Brownian Motion]
A stochastic process $\{B_t\}_{t\ge0}$ is Brownian motion if $B_0=0$, it has independent increments, $B_{t+s}-B_s\sim\mathcal{N}(0,t)$, and paths are almost surely continuous.
\end{definition}
Brownian motion, first observed by Robert Brown (1827) and mathematically formalized by Wiener (1923), is the canonical continuous-time stochastic process.

\subsection{Why Needed}
Diffusion models construct forward processes by adding Gaussian noise resembling Brownian motion. Understanding its properties allows connection to partial differential equations that describe density evolution.

\section{Itô Calculus}
\subsection{Stochastic Integrals}
For a stochastic process $X_t$ adapted to the filtration of $B_t$, the Itô integral is defined as the mean-square limit
\begin{equation}
\int_0^T X_t \, dB_t = \lim_{n\to\infty} \sum_{k=0}^{n-1} X_{t_k} (B_{t_{k+1}}-B_{t_k}).
\end{equation}

\subsection{Itô's Formula}
\begin{theorem}[Itô's Formula \cite{ito1944}]
If $X_t$ satisfies $dX_t=\mu(t,X_t)dt+\sigma(t,X_t)dB_t$ and $f\in C^{2}$, then
\begin{equation}
df(X_t) = \left(\frac{\partial f}{\partial t}+\mu\frac{\partial f}{\partial x}+\frac{1}{2}\sigma^2\frac{\partial^2 f}{\partial x^2}\right)dt + \sigma\frac{\partial f}{\partial x} dB_t.
\end{equation}
\end{theorem}
Itô calculus is indispensable for deriving diffusion and score-based models.

\section{Fokker--Planck Equation}
Let $X_t$ satisfy the stochastic differential equation (SDE)
\begin{equation}
dX_t = f(X_t,t) dt + g(X_t,t) dB_t.
\end{equation}
The probability density $p(x,t)$ of $X_t$ solves the Fokker--Planck equation
\begin{equation}
\frac{\partial p}{\partial t} = -\nabla_x\cdot (f p) + \frac{1}{2}\nabla_x^2:(gg^T p),
\end{equation}
which describes how densities evolve over time. Diffusion models design $f$ and $g$ to allow tractable reverse-time sampling.

\section{Ornstein--Uhlenbeck Process}
The Ornstein--Uhlenbeck (OU) process satisfies
\begin{equation}
dX_t = -\gamma X_t dt + \sigma dB_t,
\end{equation}
and has stationary distribution $\mathcal{N}(0, \sigma^2/(2\gamma))$. OU dynamics model velocity in Langevin equations and appear in continuous-time auto-regressive priors.

\section{Langevin Dynamics}
Langevin dynamics describe sampling from $p(x)\propto e^{-U(x)}$ via
\begin{equation}
dX_t = -\nabla U(X_t) dt + \sqrt{2} dB_t.
\end{equation}
Discretizing yields the unadjusted Langevin algorithm used for MCMC in EBMs. Adding momentum produces Hamiltonian Monte Carlo.

\section{Time Reversal of SDEs}
For an SDE $dX_t=f(X_t,t)dt+g(t)dB_t$ with density $p_t$, Anderson \cite{anderson1982} showed the time-reversed process satisfies
\begin{equation}
dX_t = \left[f(X_t,t)-g^2(t)\nabla_x \log p_t(X_t)\right]dt + g(t) d\bar{B}_t,
\end{equation}
where $\bar{B}_t$ is backward Brownian motion. This result underpins diffusion model sampling.

\section{Girsanov's Theorem}
Girsanov \cite{girsanov1960} provides a change-of-measure formula: if $X_t$ satisfies $dX_t = f dt + g dB_t$ and $u_t$ is adapted with $\int_0^T \|u_t\|^2 dt < \infty$, then under the new measure defined by
\begin{equation}
\frac{d\mathbb{Q}}{d\mathbb{P}} = \exp\left(\int_0^T u_t^T dB_t - \frac12\int_0^T \|u_t\|^2 dt\right),
\end{equation}
the process $\tilde{B}_t = B_t - \int_0^t u_s ds$ is Brownian. Girsanov's theorem allows incorporating drift changes and is used in score-based diffusion analyses.

